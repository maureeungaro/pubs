
\subsection{EC Sampling Fraction}
When going through the EC calorimeter, in the momentum range of particles detected in CLAS, 
charged pions are minimum ionizing particles, while electrons shower with a total energy 
deposition $E_{tot}$ proportional to their momentum $p$. 
Therefore the sampling fraction $E_{tot}/p$ should be independent of momentum (in reality there 
is a slight dependence).

The total energy in the calorimeter $E_{tot}$ is not always calculated to be the sum of the 
energies in the inner and outer part of the calorimeter $E_{in}$ and $E_{out}$, due to wrong 
calculation/comparison with the DC momentum \cite{bib:ectotmax}. In this analysis we
recalculated $E_{tot}$ as $E_{in}+E_{out}$ when that happened, by taking the larger
between $E_{tot}$ and $E_{in}+E_{out}$.

\begin{figure}[ht]
  \centering
		\includegraphics[width=0.9\textwidth] {img/cut-05-sf_sector-1.png}
		\includegraphics[width=0.47\textwidth]{img/slice-04_cut-05-sf_sector-1.png}
		\includegraphics[width=0.47\textwidth]{img/slice-08_cut-05-sf_sector-1.png}
		\caption{Top: Sampling Fraction as a function of momentum for Sector 1.
					Bottom: four momentum slices, and gaussian + second order 
					polynomial fit. The number of sigmas that define the
               cut are: upper: $3.4\sigma$; lower: $3\sigma$.}
 		\label{fig:sampling_fraction_s1}
\end{figure}

After applying all the other electron ID cuts, the sampling fraction is plotted in each sector
as a function of momentum (see Fig. \ref{fig:sampling_fraction_s1}). 
The plot is divided in 15 momentum slices
and each slice is fitted with a gaussian + second order polynomial function. The final result
is a $3rd$ order polynomial function that parametrizes the mean and the sigma of the 
sampling fraction as a function of $p$.
Since the negative pions in this plot would be below the electrons, the cut chosen is not exactly
symmetric around the mean, but looser on the upper part: upper: $3.4\sigma$; lower: $3\sigma$.

In Fig.~\ref{fig:sampling_fractioncut_s1} the Sampling Fraction for sector 1 is plotted for
no cuts, all other cuts, all other negative cuts and all cuts respectively. One can see 
that all the other cuts result in a quite good selection already, and that the sampling 
fraction cut (d) keeps  $\sim 90\%$ of those events.

In Fig.~\ref{fig:ecp_all_sectors} a comparison of the sampling fraction in all sectors is shown.
The cut values used in each sector and their effectiveness are summarized in 
table\,\ref{tab:sfcut}. The parameters used are listed in sec.\ref{sec:ecp_parameters}.

\begin{table}[h]
\label{tab:sfcut}
	\begin{center}
		\begin{tabular}{c | c | c | c}
			\hline 
			\multirow{2}{*}{Sector} 
					& events with EC & SF cut\\
					&  GeV & \% & \% \\
			\hline
			1   & 73.8 & 60.1 \\
			2   & 74.7 & 59.0 \\
			3   & 75.6 & 57.5 \\
			4   & 73.0 & 58.5 \\
			5   & 74.9 & 60.3 \\
			6   & 75.0 & 58.5 \\
			\hline 
		\end{tabular}
		\caption{The Sampling Fraction (SF) cut values and effectiveness in each sector.
					The second column refers to events with signal in EC that pass the SF cut.}	
	\end{center}
\end{table}


\begin{figure}[ht]
  \centering
		\includegraphics[width=0.98\textwidth]{img/cut-05-sfc_sector-1.png}
		\caption{Sampling Fraction cut. Notice in panel (b) all the other cuts
        but the sampling fraction applied; the sampling fraction cut (d) keeps
		$90.5\,^{\circ\!\!}/\!_\circ$ of the events in panel (b).}
 		\label{fig:sampling_fractioncut_s1}
\end{figure}

\clearpage

\begin{figure}[ht]
  \centering
		\includegraphics[width=0.9\textwidth]{img/cut-05-sf_sector-all.png}
		\caption{Sampling Fraction cut in all sectors. Plot grid and 
					the line at 0.3 emphasize differences between sectors.}
 		\label{fig:ecp_all_sectors}
\end{figure}

\subsubsection{Cut parameters}\label{sec:ecp_parameters}
$$
f(p) = a + bp + cp^2 + dp^3
$$
\begin{verbatim}
         S1           S2           S3           S4          S5            S6
mean:
a:     0.249471     0.252591     0.250881     0.246831     0.247017     0.248919
b:    0.0350377    0.0487858    0.0443294    0.0381502    0.0326401    0.0467523
c:  -0.00887004   -0.0130587   -0.0101895  -0.00993875  -0.00788004   -0.0110489
d:  0.000827307   0.00120111  0.000785888  0.000882525  0.000684981  0.000902679

sigma 
a:    0.0482422    0.0501838    0.0483883    0.0435729    0.0463011    0.0452695
b:   -0.0234008   -0.0206365   -0.0243889    -0.018439   -0.0173496   -0.0180503
c:   0.00652057   0.00521743   0.00709145   0.00486129   0.00460028   0.00473553
d: -0.000643782 -0.000504944 -0.000731488 -0.000498339  -0.00041353 -0.000477848
\end{verbatim}
