\clearpage\newpage

\subsection{CC $\phi$ Matching}
The principle of this cut is very simple: when the track is on the right of the CC, the right
photo-multiplier should fire, and viceversa. Exception: when $\phi$ (relative to the center
in each sector) is less than $4^0$ the track is kept (the \v Cerenkov light should hit both
pmts, but with less efficiency since it splits in the middle)\cite{bib:ccmatch},\cite{bib:pc_fxpun}, \cite{bib:pc_osi}.

To show the effects of this cut the quantity ``$\phi$ matching'' is plotted in
Fig.~\ref{fig:ccm_phi}. This quantity is $0$ when both pmts are fired, $1(-1)$ when
there is a left (right) match and $2(-2)$ there is a left (right) mismatch.
The cut applied is:  ``$\phi$ matching''$<2$ except when $|\phi|<4^0$.

\begin{figure}[ht]
  \centering
		\includegraphics[width=0.93\textwidth]{img/cut-02-cc-phi-match_sector-1.png}
		\caption{``$\phi$ matching``: this quantity is $0$ when both pmts are fired, $1(-1)$ when
there is a left (right) match and $2(-2)$ there is a left (right) mismatch.
The cut applied is:  ``$\phi$ matching''$<2$ except when $|\phi|<4^0$.}
 		\label{fig:ccm_phi}
\end{figure}

