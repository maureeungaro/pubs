\clearpage\newpage


\subsection{Electromagnetic Shower Shape}
Due to their shower shape, electrons release more energy in 
the inner part of the calorimeter than in the outer part. In
fact the energy released in the inner part constitutes a good
fraction of the total energy in the calorimeter.
We chose to keep the events with $$E_{in}/E_{TOT} \geq 0.25$$

This cut is very loose since most of the MIP are already cut
out with all the other cuts, ass seen in Fig.~\ref{fig:einetot}.
This cut keeps more than $99\%$ of events in each sector.


\begin{figure}[ht]
  \centering
		\includegraphics[width=0.95\textwidth]{img/cut-11EoOEtot_sector-1.png}
		\caption{$E_{in}/E_{TOT}$ for Sector 1. All other cuts (panel b.)
					almost completely rid of the MIP: $99.1\%$ of those events are kept (Panel d.). 
					Panel c. shows all other negative cuts except the minimum $p$ cut.}
 		\label{fig:einetot}
\end{figure}

