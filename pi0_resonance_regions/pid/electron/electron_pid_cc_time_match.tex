\clearpage\newpage

\subsection{CC Time Matching}
The CC timing was not calibrated in e1-6, but a timing cut is still possible if applied to each tube
(this is basically equivalent to perform the timing calibration).

The difference $\Delta T$ between the track time recorded on a CC segment and corresponding time recorded on the TOF,
corrected for the path length from the CC to the TOF, is fitted with a gaussian (see Fig.~\ref{fig:cc_time_slices}).
Since there could be multiple \v Cerenkov light reflections leading to a time delay,
a 3 sigma cut is applied on the {\it left} of the signal, and not on the right.
This difference is plotted in Fig.~\ref{fig:cc_time_sec1} for all tubes in sector one.

\begin{figure}[ht]
  \centering
		\includegraphics[width=0.46\textwidth]{img/slice-03_cut-03-cc-time-match_sector-1.png}
		\includegraphics[width=0.46\textwidth]{img/slice-08_cut-03-cc-time-match_sector-1.png}
		\includegraphics[width=0.46\textwidth]{img/slice-12_cut-03-cc-time-match_sector-1.png}
		\includegraphics[width=0.46\textwidth]{img/slice-15_cut-03-cc-time-match_sector-1.png}
		\caption{CC time matching. The difference $\Delta T$ between the track time recorded
               at a CC tube ($T_{CC}$) and corresponding time recorded on the TOF ($T_{SC}$),
               corrected for the path length from the CC to the TOF ($|\vec{R_{CC}}-\vec{R_{SC}}|/c$),
               shown here for 4 CC pmts, is fitted with a gaussian.
               A 5$\sigma$ cut is applied on the left of the signal.}
 		\label{fig:cc_time_slices}
\end{figure}

\begin{figure}[ht]
  \centering
		\includegraphics[width=0.98\textwidth]{img/cut-03cctimd_sector-1.png}
		\caption{CC time matching. The difference $\Delta T$ between the track time recorded
               at a CC tube ($T_{CC}$) and corresponding time recorded on the TOF ($T_{SC}$),
               corrected for the path length from the CC to the TOF ($|\vec{R_{CC}}-\vec{R_{SC}}|/c$),
               is fitted with a gaussian. A 3$\sigma$ cut is applied on the left of the signal.
               Top left: all events. Top right: events with calorimeter cuts applied.
               notice that these cuts remove $71 \,^{\circ\!\!}/\!_\circ$ of the data.
               Bottom left: events with the negative calorimeter cuts applied.
               Bottom right: all cuts applied. Notice that the CC matching cut
               only removes $7  \,^{\circ\!\!}/\!_\circ$ of the events with
               the calorimeter cuts already applied.}
 		\label{fig:cc_time_sec1}
\end{figure}

