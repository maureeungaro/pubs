\clearpage\newpage
\subsection{EC Threshold}
A study \cite{bib:ecmin} of the inclusive cross section at various beam energies in CLAS 
results in a parametrization of the low momentum cut $p_{min}$ as a function of
the calorimeter low total threshold (in milliVolts) of the trigger discriminator:
\begin{equation}
 \label{eq:pmin} 
 p_{min}\,\,{\rm (MeV)} = 214 + 2.47\times EC_{threshold}{\rm (mV)}
\end{equation}

The low total threshold for e1-6 was $172$ mV therefore the minimum momentum cut is fixed at:
$$
p_{min} = 0.64\,\,{\rm GeV}
$$

Fig.~\ref{fig:pmincut_alls} shows for the momentum distribution of the candidates integrated
over all sectors. In average, $\sim 27.7\%$  pass the all other particle ID
cuts and of these, $91.9\%$ pass the minimum $p$ cut.

The cut value used is the same for all sectors and its effectiveness is summarized in 
table\,\ref{tab:pmincut}.


\begin{figure}[ht]
  \centering
		\includegraphics[width=0.88\textwidth]{img/cut-04-ec-threshold_sector-1.png}
		\caption{Candidates Momentum distribution in each sector. The minimum momentum cut is
               chosen according to Eq.\ref{eq:pmin}. In average, $\sim 82 \,^{\circ\!\!}/\!_\circ$ 
					of the candidates have a signal in the EC. Of those, $30 \,^{\circ\!\!}/\!_\circ$
					pass the all other particle ID cuts and of these, $92.5 \,^{\circ\!\!}/\!_\circ$
					pass the minimum $p$ cut.}
 		\label{fig:pmincut_alls}
\end{figure}

\clearpage



\begin{table}[h]
\label{tab:pmincut}
	\begin{center}
		\begin{tabular}{c | c | c | c}
			\hline 
			\multirow{2}{*}{Sector} 
					& all other cuts & minimum $p$ cut \\
					&  GeV & \% &  \\
			\hline 
			1   & 71.1 & 93.1 \\
			2   & 72.1 & 89.8 \\
			3   & 71.9 & 91.6 \\
			4   & 71.8 & 93.7 \\
			5   & 75.2 & 90.0 \\
			6   & 72.6 & 92.8 \\
			\hline
		\end{tabular}
		\caption{The minimum $p$ cut values and effectiveness in each sector.
					The last column refers to events with signal in EC that pass the 
 					minimum $p$ cut.}	
	
	\end{center}
\end{table}


