\clearpage\newpage
\subsection{EC Threshold}
A study \cite{bib:ecmin} of the inclusive cross section at various beam energies in CLAS 
results in a parametrization of the low momentum cut $p_{min}$ as a function of
the calorimeter low total threshold (in milliVolts) of the trigger discriminator:
\begin{equation}
 \label{eq:pmin} 
 p_{min}\,\,{\rm (MeV)} = 214 + 2.47\times EC_{threshold}{\rm (mV)}
\end{equation}

The low total threshold for e1-6 was $172$ mV therefore the minimum momentum cut is fixed at:
$$
p_{min} = 0.64\,\,{\rm GeV}
$$

Fig.~\ref{fig:pmincut_alls} shows for the momentum distribution of the candidates integrated
over all sectors. In average, $\sim 27.7\%$  pass the all other particle ID
cuts and of these, $91.9\%$ pass the minimum $p$ cut.

The cut value used is the same for all sectors and its effectiveness is summarized in 
table\,\ref{tab:pmincut}.


\begin{figure}[ht]
  \centering
		\includegraphics[width=0.88\textwidth]{img/cut-04pthr_sector-all.png}
		\caption{Candidates Momentum distribution in each sector. The minimum momentum cut is
               chosen according to Eq.\ref{eq:pmin}. In average, $\sim 82 \,^{\circ\!\!}/\!_\circ$ 
					of the candidates have a signal in the EC. Of those, $30 \,^{\circ\!\!}/\!_\circ$
					pass the all other particle ID cuts and of these, $92.5 \,^{\circ\!\!}/\!_\circ$
					pass the minimum $p$ cut.}
 		\label{fig:pmincut_alls}
\end{figure}

\clearpage



\begin{table}[h]
\label{tab:pmincut}
	\begin{center}
		\begin{tabular}{c | c | c | c}
			\hline 
			\multirow{2}{*}{Sector} 
					& all other cuts & minimum $p$ cut \\
					&  GeV & \% &  \\
			\hline 
			1   & 71.1 & 93.1 \\
			2   & 72.1 & 89.8 \\
			3   & 71.9 & 91.6 \\
			4   & 71.8 & 93.7 \\
			5   & 75.2 & 90.0 \\
			6   & 72.6 & 92.8 \\
			\hline
		\end{tabular}
		\caption{The minimum $p$ cut values and effectiveness in each sector.
					The last column refers to events with signal in EC that pass the 
 					minimum $p$ cut.}	
	
	\end{center}
\end{table}

\subsection{EC Sampling Fraction}
When going through the EC calorimeter, in the momentum range of particles detected in CLAS, 
charged pions are minimum ionizing particles, while electrons shower with a total energy 
deposition $E_{tot}$ proportional to their momentum $p$. 
Therefore the sampling fraction $E_{tot}/p$ should be independent of momentum (in reality there 
is a slight dependence).

The total energy in the calorimeter $E_{tot}$ is not always calculated to be the sum of the 
energies in the inner and outer part of the calorimeter $E_{in}$ and $E_{out}$, due to wrong 
calculation/comparison with the DC momentum \cite{bib:ectotmax}. In this analysis we
recalculated $E_{tot}$ as $E_{in}+E_{out}$ when that happened, by taking the larger
between $E_{tot}$ and $E_{in}+E_{out}$.

\begin{figure}[ht]
  \centering
		\includegraphics[width=0.9\textwidth] {img/cut-05sampf_sector-1.png}
		\includegraphics[width=0.47\textwidth]{img/slice-04_cut-05sampf_sector-1.png}
		\includegraphics[width=0.47\textwidth]{img/slice-08_cut-05sampf_sector-1.png}
		\caption{Top: Sampling Fraction as a function of momentum for Sector 1.
					Bottom: four momentum slices, and gaussian + second order 
					polynomial fit. The number of sigmas that define the
               cut are: upper: $3.4\sigma$; lower: $3\sigma$.}
 		\label{fig:sampling_fraction_s1}
\end{figure}

After applying all the other electron ID cuts, the sampling fraction is plotted in each sector
as a function of momentum (see Fig. \ref{fig:sampling_fraction_s1}). 
The plot is divided in 15 momentum slices
and each slice is fitted with a gaussian + second order polynomial function. The final result
is a $3rd$ order polynomial function that parametrizes the mean and the sigma of the 
sampling fraction as a function of $p$.
Since the negative pions in this plot would be below the electrons, the cut chosen is not exactly
symmetric around the mean, but looser on the upper part: upper: $3.4\sigma$; lower: $3\sigma$.

In Fig.~\ref{fig:sampling_fractioncut_s1} the Sampling Fraction for sector 1 is plotted for
no cuts, all other cuts, all other negative cuts and all cuts respectively. One can see 
that all the other cuts result in a quite good selection already, and that the sampling 
fraction cut (d) keeps  $\sim 90\%$ of those events.

In Fig.~\ref{fig:ecp_all_sectors} a comparison of the sampling fraction in all sectors is shown.
The cut values used in each sector and their effectiveness are summarized in 
table\,\ref{tab:sfcut}. The parameters used are listed in sec.\ref{sec:ecp_parameters}.

\begin{table}[h]
\label{tab:sfcut}
	\begin{center}
		\begin{tabular}{c | c | c | c}
			\hline 
			\multirow{2}{*}{Sector} 
					& events with EC & SF cut\\
					&  GeV & \% & \% \\
			\hline
			1   & 73.8 & 60.1 \\
			2   & 74.7 & 59.0 \\
			3   & 75.6 & 57.5 \\
			4   & 73.0 & 58.5 \\
			5   & 74.9 & 60.3 \\
			6   & 75.0 & 58.5 \\
			\hline 
		\end{tabular}
		\caption{The Sampling Fraction (SF) cut values and effectiveness in each sector.
					The second column refers to events with signal in EC that pass the SF cut.}	
	\end{center}
\end{table}


\begin{figure}[ht]
  \centering
		\includegraphics[width=0.98\textwidth]{img/cut-05sampfd_sector-1.png}
		\caption{Sampling Fraction cut. One can see in panel (b) that all the other cuts 
		result in a quite good selection already, and that the sampling fraction cut (d) keeps  
		$90.5\,^{\circ\!\!}/\!_\circ$ of those events.}
 		\label{fig:sampling_fractioncut_s1}
\end{figure}

\clearpage\newpage
\begin{figure}[ht]
  \centering
		\includegraphics[width=0.8\textwidth]{img/cut-05sampf_sector-all.png}
		\caption{Sampling Fraction cut in all sectors. Plot grid and 
					the line at 0.3 emphasize differences between sectors.}
 		\label{fig:ecp_all_sectors}
\end{figure}

\subsubsection{Cut parameters}\label{sec:ecp_parameters}
$$
f(p) = a + bp + cp^2 + dp^3
$$
\begin{verbatim}   

         S1           S2           S3           S4          S5            S6
mean:
a:     0.249471     0.252591     0.250881     0.246831     0.247017     0.248919
b:    0.0350377    0.0487858    0.0443294    0.0381502    0.0326401    0.0467523
c:  -0.00887004   -0.0130587   -0.0101895  -0.00993875  -0.00788004   -0.0110489
d:  0.000827307   0.00120111  0.000785888  0.000882525  0.000684981  0.000902679

sigma 
a:    0.0482422    0.0501838    0.0483883    0.0435729    0.0463011    0.0452695
b:   -0.0234008   -0.0206365   -0.0243889    -0.018439   -0.0173496   -0.0180503
c:   0.00652057   0.00521743   0.00709145   0.00486129   0.00460028   0.00473553
d: -0.000643782 -0.000504944 -0.000731488 -0.000498339  -0.00041353 -0.000477848
\end{verbatim}

\clearpage\newpage
\subsection{Track Coordinates in the EC plane}
The EC is designed for the electron to release all their energy in it.
However electrons that shower near the edges of the calorimeter will not loose
all their energy in the detector because the shower is not fully contained,
thus their energy cannot be properly reconstructed. For this reason 
a fiducial cut is introduced on the track coordinates $U,V,W$ 
of the electrons at the EC plane. The $U,V,W$ coordinates are chosen for 
convenience since they are parallel to the directions of the EC scintillators 
(and to the EC edges). The cuts have been adjusted by looking at Fig.~\ref{fig:ccm_phi}
and making sure the distribution is $\phi$-symmetric
$$
 40\leq U\leq400, V\leq362, W\leq395
$$
The $U,V,W$ distributions are plotted in Fig.~\ref{fig:ECu},\ref{fig:ECv},\ref{fig:ECw},
respectively. In average $80.9\%$ of all the events with an EC signal pass the U cut, $71.3\%$ the V cut
and $69.4\%$ the W cut, for a combined pass/total ratio of $58\%$.  When all other cuts are applied
the U,V,W keep $96.6\%$, $88.6\%$, $86.8\%$ events respectively, for a combined effective cut of $77.8\%$
(the product of these numbers is  $74.2\%$ but the corners events are correlated).

In Fig.~\ref{fig:ECyx} is plotted the Y versus X track coordinate in the EC plane before and after
the $U,V,W$ cuts.



\clearpage
\begin{figure}[ht]
  \centering
		\includegraphics[width=0.7\textheight]{img/cut-06ECU_sector-all.png}
		\caption{U track coordinate in the EC plane for all sectors. The cut is chosen to avoid
              edge effects that truncate the electron shower.}
 		\label{fig:ECu}
\end{figure}
\clearpage

\clearpage
\begin{figure}[ht]
  \centering
		\includegraphics[width=0.7\textheight]{img/cut-07ECV_sector-all.png}
		\caption{V track coordinate in the EC plane for all sectors. The cut is chosen to avoid
              edge effects that truncate the electron shower.}
 		\label{fig:ECv}
\end{figure}

\clearpage
\begin{figure}[ht]
  \centering
		\includegraphics[width=0.7\textheight]{img/cut-08ECW_sector-all.png}
		\caption{W track coordinate in the EC plane for all sectors. The cut is chosen to avoid
              edge effects that truncate the electron shower.}
 		\label{fig:ECw}
\end{figure}

\begin{figure}[ht]
  \centering
		\includegraphics[width=0.98\textwidth]{img/cut-09uvw_sector-all.png}
		\caption{Y versus X track coordinate in the EC plane before and after 
					the $U,V,W$ cuts.}
 		\label{fig:ECyx}
\end{figure}


\clearpage
\subsection{Minimum Ionizing Particles (MIP) rejection}
The outer EC is $5/3$ times bigger than the inner EC. Therefore pions,
which do not shower and are minimum ionizing particles in the momentum range 
detected in CLAS, release a (small) quantity of energy in the outer and inner
parts in the ratio $5:3$, independent on their momentum.

In Fig.~\ref{fig:EoEi} the $E_{out}/p$ versus $E_{in}/p$ is shown for Sector 1.
One can see the MIP signal along the $y=5/3x $ line.
Panel b. shows the same quantity when all other cuts are applied. 
One can see that the electron signal on the right.
The cut is extrapolated by visually comparing panel a. and b. and trying to
cut the most MIP as possible with a straight line $y = a + bx$. The line
is sector dependent as shown in Fig.~\ref{fig:EoEi_all}.
This cut also include a minimum EC outer energy requirement of $1MeV$.
The cut values used in each sector and their effectiveness are summarized in 
table\,\ref{tab:EoEi}.


\begin{figure}[ht]
  \centering
		\includegraphics[width=0.93\textwidth]{img/cut-10EoVsEi_sector-1.png}
		\caption{$E_{out}/p$ versus $E_{in}/p$ for Sector 1.}
 		\label{fig:EoEi}
\end{figure}


\clearpage\newpage
\begin{figure}[ht]
  \centering
		\includegraphics[width=1.00\textwidth]{img/cut-10EoVsEi_sector-all.png}
		\caption{$E_{out}/p$ versus $E_{in}/p$ for all sectors. In every sector
					roughly $99\,^{\circ\!\!}/\!_\circ$ of events with all other cuts applied also pass this cut. }
 		\label{fig:EoEi_all}
\end{figure}

\begin{table}[ht]
\label{tab:EoEi}
	\begin{center}
		\begin{tabular}{c | c | c | c}
			\hline 
			\multirow{2}{*}{Sector} 
					& y = a + bx pars & MIP cut / All others \\
					&   & \% & \% \\
			\hline
			1    & $a=0.32$, $b=-2.2$ & 98.7 \\
			2    & $a=0.32$, $b=-2.2$ & 98.6 \\
			3    & $a=0.34$, $b=-2.2$ & 98.5 \\
			4    & $a=0.31$, $b=-2.1$ &  99.1 \\
			5    & $a=0.30$, $b=-2.2$ &  98.5 \\
			6    & $a=0.33$, $b=-2.1$ &  98.3 \\
			\hline
		\end{tabular}
		\caption{The Minimum Ionizing cut values and effectiveness in each sector.
					The last column refers to events with signal in EC that pass the MIP cut.}	
	\end{center}
\end{table}





\subsection{Electromagnetic Shower Shape}
Due to their shower shape, electrons release more energy in 
the inner part of the calorimeter than in the outer part. In
fact the energy released in the inner part constitutes a good
fraction of the total energy in the calorimeter.
We chose to keep the events with $$E_{in}/E_{TOT} \geq 0.25$$

This cut is very loose since most of the MIP are already cut
out with all the other cuts, ass seen in Fig.~\ref{fig:einetot}.
This cut keeps more than $99\%$ of events in each sector.


\begin{figure}[ht]
  \centering
		\includegraphics[width=0.95\textwidth]{img/cut-11EoOEtot_sector-1.png}
		\caption{$E_{in}/E_{TOT}$ for Sector 1. All other cuts (panel b.)
					almost completely rid of the MIP: $99.1\%$ of those events are kept (Panel d.). 
					Panel c. shows all other negative cuts except the minimum $p$ cut.}
 		\label{fig:einetot}
\end{figure}


\clearpage\newpage
\subsection{Number of photo-electrons in the \v Cerenkov detector}
\label{sec:cc_cut}
In the past a threshold for the signal in the \v Cerenkov detector was necessary to eliminate
electronic noise and the fact that negative pions produce \v Cerenkov light when 
their momentum is above $\sim 2.5$ GeV.

The ADC signal from the CC is converted in 
{\it number of photo-electrons} (nphe) and
multiplied by 10. The number of photo-electrons detected in the \v Cerenkov
Counter for electrons is typically between 5 and 20, or $10\times nphe_{el} \sim 50-200$.
You can see from Fig.\,\ref{fig:cccut_alls} the $10\times nphe$
distribution in each sector.

The peak at $nphe \sim 1$ represents not only background and
negative pions, but good electrons with low CC efficency hits. For this reason
it's better to apply the CC $\theta$, $\phi$ and Timing match cuts.

In average (see Fig.~\ref{fig:cccut_alls}) only $\sim 25\%$ of candidates pass the calorimeter cuts. 
If applied, the nphe cut would keep $\sim 90\%$ of them.

The $nphe$ cut is chosen visually to be at the minimum
of the $nphe$ distribution between the background and electron signals peak.


\begin{figure}[ht]
  \centering
		\includegraphics[width=0.85\textwidth ]{img/cut-12nphe_sector-all.png}
		\caption{10 $\times$ Number of photo-electrons distribution integrated over all sectors.
               In average, only $\sim 25 \,^{\circ\!\!}/\!_\circ$ pass the calorimeter cuts and of these, 
               $\sim 90 \,^{\circ\!\!}/\!_\circ$ pass the $nphe$ cut.}
 		\label{fig:cccut_alls}
\end{figure}
		




\subsection{Summary of Electron Identification }
In Fig.~\ref{fig:epidsummary} a summary of the electron identifications cuts in all sectors is shown:

\begin{itemize}
	\item Individual cuts effect: this is the percentage of events that would be selected by each
	individual cut if it was the only cut applied.
	\item Effectiveness: this is the percentage of events that are selected by each individual cut when
	all the other cuts are effective.
	\item Cumulative: this is the percentage of events that are selected by each individual cut when
	they are applied in the order show.
\end{itemize}


\begin{figure}[hb]
  \centering
		\includegraphics[width=1.05\textwidth]{img/epidsummary.png}
		\caption{Electron Identification Summary. Top: effects of individual cuts.
		         Middle: effectiveness of each cut.
               Bottom: effects of accumulated (ordered) cuts.}
 		\label{fig:epidsummary}
\end{figure}



\clearpage\newpage










