\clearpage\newpage

\subsection{Minimum Ionizing Particles (MIP) rejection}
The outer EC is $5/3$ times bigger than the inner EC. Therefore pions,
which do not shower and are minimum ionizing particles in the momentum range 
detected in CLAS, release a (small) quantity of energy in the outer and inner
parts in the ratio $5:3$, independent on their momentum.

In Fig.~\ref{fig:EoEi} the $E_{out}/p$ versus $E_{in}/p$ is shown for Sector 1.
One can see the MIP signal along the $y=5/3x $ line.
Panel b. shows the same quantity when all other cuts are applied. 
One can see that the electron signal on the right.
The cut is extrapolated by visually comparing panel a. and b. and trying to
cut the most MIP as possible with a straight line $y = a + bx$. The line
is sector dependent as shown in Fig.~\ref{fig:EoEi_all}.
This cut also include a minimum EC outer energy requirement of $1MeV$.
The cut values used in each sector and their effectiveness are summarized in 
table\,\ref{tab:EoEi}.


\begin{figure}[ht]
  \centering
		\includegraphics[width=0.93\textwidth]{img/cut-10-mip_sector-1.png}
		\caption{$E_{out}/p$ versus $E_{in}/p$ for Sector 1.}
 		\label{fig:EoEi}
\end{figure}


\clearpage\newpage
\begin{figure}[ht]
  \centering
		\includegraphics[width=1.00\textwidth]{img/cut-10-mip_sector-all.png}
		\caption{$E_{out}/p$ versus $E_{in}/p$ for all sectors. In every sector
					roughly $99\,^{\circ\!\!}/\!_\circ$ of events with all other cuts applied also pass this cut. }
 		\label{fig:EoEi_all}
\end{figure}

\begin{table}[ht]
\label{tab:EoEi}
	\begin{center}
		\begin{tabular}{c | c | c | c}
			\hline 
			\multirow{2}{*}{Sector} 
					& y = a + bx pars & MIP cut / All others \\
					&   & \% & \% \\
			\hline
			1    & $a=0.32$, $b=-2.2$ & 98.7 \\
			2    & $a=0.32$, $b=-2.2$ & 98.6 \\
			3    & $a=0.34$, $b=-2.2$ & 98.5 \\
			4    & $a=0.31$, $b=-2.1$ &  99.1 \\
			5    & $a=0.30$, $b=-2.2$ &  98.5 \\
			6    & $a=0.33$, $b=-2.1$ &  98.3 \\
			\hline
		\end{tabular}
		\caption{The Minimum Ionizing cut values and effectiveness in each sector.
					The last column refers to events with signal in EC that pass the MIP cut.}	
	\end{center}
\end{table}



