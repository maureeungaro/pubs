\clearpage
\subsection{Minimum Ionizing Particles (MIP) rejection}
The outer EC is $5/3$ times bigger than the inner EC. Therefore pions,
which do not shower and are minimum ionizing particles in the momentum range 
detected in CLAS, release a (small) quantity of energy in the outer and inner
parts in the ratio $5:3$, independent on their momentum.

In Fig.~\ref{fig:EoEi} the $E_{out}/p$ versus $E_{in}/p$ is shown for Sector 1.
One can see the MIP signal along the $y=5/3x $ line.
Panel b. shows the same quantity when all other cuts are applied. 
One can see that the electron signal on the right.
The cut is extrapolated by visually comparing panel a. and b. and trying to
cut the most MIP as possible with a straight line $y = a + bx$. The line
is sector dependent as shown in Fig.~\ref{fig:EoEi_all}.
This cut also include a minimum EC outer energy requirement of $1MeV$.
The cut values used in each sector and their effectiveness are summarized in 
table\,\ref{tab:EoEi}.


\begin{figure}[ht]
  \centering
		\includegraphics[width=0.93\textwidth]{img/cut-10EoVsEi_sector-1.png}
		\caption{$E_{out}/p$ versus $E_{in}/p$ for Sector 1.}
 		\label{fig:EoEi}
\end{figure}


\clearpage\newpage
\begin{figure}[ht]
  \centering
		\includegraphics[width=1.00\textwidth]{img/cut-10EoVsEi_sector-all.png}
		\caption{$E_{out}/p$ versus $E_{in}/p$ for all sectors. In every sector
					roughly $99\,^{\circ\!\!}/\!_\circ$ of events with all other cuts applied also pass this cut. }
 		\label{fig:EoEi_all}
\end{figure}

\begin{table}[ht]
\label{tab:EoEi}
	\begin{center}
		\begin{tabular}{c | c | c | c}
			\hline 
			\multirow{2}{*}{Sector} 
					& y = a + bx pars & MIP cut / All others \\
					&   & \% & \% \\
			\hline
			1    & $a=0.32$, $b=-2.2$ & 98.7 \\
			2    & $a=0.32$, $b=-2.2$ & 98.6 \\
			3    & $a=0.34$, $b=-2.2$ & 98.5 \\
			4    & $a=0.31$, $b=-2.1$ &  99.1 \\
			5    & $a=0.30$, $b=-2.2$ &  98.5 \\
			6    & $a=0.33$, $b=-2.1$ &  98.3 \\
			\hline
		\end{tabular}
		\caption{The Minimum Ionizing cut values and effectiveness in each sector.
					The last column refers to events with signal in EC that pass the MIP cut.}	
	\end{center}
\end{table}





\subsection{Electromagnetic Shower Shape}
Due to their shower shape, electrons release more energy in 
the inner part of the calorimeter than in the outer part. In
fact the energy released in the inner part constitutes a good
fraction of the total energy in the calorimeter.
We chose to keep the events with $$E_{in}/E_{TOT} \geq 0.25$$

This cut is very loose since most of the MIP are already cut
out with all the other cuts, ass seen in Fig.~\ref{fig:einetot}.
This cut keeps more than $99\%$ of events in each sector.


\begin{figure}[ht]
  \centering
		\includegraphics[width=0.95\textwidth]{img/cut-11EoOEtot_sector-1.png}
		\caption{$E_{in}/E_{TOT}$ for Sector 1. All other cuts (panel b.)
					almost completely rid of the MIP: $99.1\%$ of those events are kept (Panel d.). 
					Panel c. shows all other negative cuts except the minimum $p$ cut.}
 		\label{fig:einetot}
\end{figure}


\clearpage\newpage
\subsection{Number of photo-electrons in the \v Cerenkov detector}
\label{sec:cc_cut}
In the past a threshold for the signal in the \v Cerenkov detector was necessary to eliminate
electronic noise and the fact that negative pions produce \v Cerenkov light when 
their momentum is above $\sim 2.5$ GeV.

The ADC signal from the CC is converted in 
{\it number of photo-electrons} (nphe) and
multiplied by 10. The number of photo-electrons detected in the \v Cerenkov
Counter for electrons is typically between 5 and 20, or $10\times nphe_{el} \sim 50-200$.
You can see from Fig.\,\ref{fig:cccut_alls} the $10\times nphe$
distribution in each sector.

The peak at $nphe \sim 1$ represents not only background and
negative pions, but good electrons with low CC efficency hits. For this reason
it's better to apply the CC $\theta$, $\phi$ and Timing match cuts.

In average (see Fig.~\ref{fig:cccut_alls}) only $\sim 25\%$ of candidates pass the calorimeter cuts. 
If applied, the nphe cut would keep $\sim 90\%$ of them.

The $nphe$ cut is chosen visually to be at the minimum
of the $nphe$ distribution between the background and electron signals peak.


\begin{figure}[ht]
  \centering
		\includegraphics[width=0.85\textwidth ]{img/cut-12nphe_sector-all.png}
		\caption{10 $\times$ Number of photo-electrons distribution integrated over all sectors.
               In average, only $\sim 25 \,^{\circ\!\!}/\!_\circ$ pass the calorimeter cuts and of these, 
               $\sim 90 \,^{\circ\!\!}/\!_\circ$ pass the $nphe$ cut.}
 		\label{fig:cccut_alls}
\end{figure}
		




\subsection{Summary of Electron Identification }
In Fig.~\ref{fig:epidsummary} a summary of the electron identifications cuts in all sectors is shown:

\begin{itemize}
	\item Individual cuts effect: this is the percentage of events that would be selected by each
	individual cut if it was the only cut applied.
	\item Effectiveness: this is the percentage of events that are selected by each individual cut when
	all the other cuts are effective.
	\item Cumulative: this is the percentage of events that are selected by each individual cut when
	they are applied in the order show.
\end{itemize}


\begin{figure}[hb]
  \centering
		\includegraphics[width=1.05\textwidth]{img/epidsummary.png}
		\caption{Electron Identification Summary. Top: effects of individual cuts.
		         Middle: effectiveness of each cut.
               Bottom: effects of accumulated (ordered) cuts.}
 		\label{fig:epidsummary}
\end{figure}













