
\clearpage\newpage

\subsection{Number of photo-electrons in the \v Cerenkov detector}
\label{sec:cc_cut}
In the past a threshold for the signal in the \v Cerenkov detector was necessary to eliminate
electronic noise and the fact that negative pions produce \v Cerenkov light when 
their momentum is above $\sim 2.5$ GeV.

The ADC signal from the CC is converted in 
{\it number of photo-electrons} (nphe) and
multiplied by 10. The number of photo-electrons detected in the \v Cerenkov
Counter for electrons is typically between 5 and 20, or $10\times nphe_{el} \sim 50-200$.
You can see from Fig.\,\ref{fig:cccut_alls} the $10\times nphe$
distribution in each sector.

The peak at $nphe \sim 1$ represents not only background and
negative pions, but good electrons with low CC efficency hits. For this reason
it's better to apply the CC $\theta$, $\phi$ and Timing match cuts.

In average (see Fig.~\ref{fig:cccut_alls}) only $\sim 25\%$ of candidates pass the calorimeter cuts. 
If applied, the nphe cut would keep $\sim 90\%$ of them.

The $nphe$ cut is chosen visually to be at the minimum
of the $nphe$ distribution between the background and electron signals peak.


\begin{figure}[ht]
  \centering
		\includegraphics[width=0.85\textwidth ]{img/cut-12nphe_sector-all.png}
		\caption{10 $\times$ Number of photo-electrons distribution integrated over all sectors.
               In average, only $\sim 25 \,^{\circ\!\!}/\!_\circ$ pass the calorimeter cuts and of these, 
               $\sim 90 \,^{\circ\!\!}/\!_\circ$ pass the $nphe$ cut.}
 		\label{fig:cccut_alls}
\end{figure}
		




\subsection{Summary of Electron Identification }
In Fig.~\ref{fig:epidsummary} a summary of the electron identifications cuts in all sectors is shown:

\begin{itemize}
	\item Individual cuts effect: this is the percentage of events that would be selected by each
	individual cut if it was the only cut applied.
	\item Effectiveness: this is the percentage of events that are selected by each individual cut when
	all the other cuts are effective.
	\item Cumulative: this is the percentage of events that are selected by each individual cut when
	they are applied in the order show.
\end{itemize}


\begin{figure}[hb]
  \centering
		\includegraphics[width=1.05\textwidth]{img/epidsummary.png}
		\caption{Electron Identification Summary. Top: effects of individual cuts.
		         Middle: effectiveness of each cut.
               Bottom: effects of accumulated (ordered) cuts.}
 		\label{fig:epidsummary}
\end{figure}













