\abstract{

    GEMC is a Monte-Carlo simulation program that uses a database-driven approach to build
    all experiment parameters. This includes, but not limited to, the detector geometry,
    materials, digitization, and readout electronics.

    The framework is experiment independent: it allows to build any setup
    from a database source (such as MYSQL, CSQL, TEXT) w/o the need of recompiling the code.

    The engine is written in C++ and uses Geant4 for the passage of particles through matter.

    It includes a python API to populate the databases and provides all geant4 functionality.
    In addition, it handles real-life scenarios like geometry variations and run number
    dependency of quantities like calibration constants and digitization parameters.

    An overview of the software, and its usage will be shown, with examples on how to build geometry, handle geometry variations, and provide realistic electronic response.

    The usage of GEMC at Jefferson Lab in the CLAS12 experimental program will be showcased.
}