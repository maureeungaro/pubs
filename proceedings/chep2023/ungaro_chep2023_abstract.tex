\abstract{
    GEMC is an application that harnesses the power of databases
    to execute Geant4 Monte-Carlo simulations.

    The databases (MYSQL, CSQL, TEXT) define the geometry,
    materials, digitization algorithms, readout electronics
    and output formats.

    Implemented in C++, GEMC also boasts a user-friendly Python API that
    facilitates detector construction and database population.

    What sets GEMC apart is its capability to handle real-life scenarios,
    accommodating geometry variations and the run number-dependent calibration
    constants and digitization parameters within its framework.

    This abstract provides an overview of GEMC, accompanied by illustrative examples
    that showcase its versatility. We delve into the practical application
    of GEMC within the the CLAS12 experimental program at Jefferson Lab.

}