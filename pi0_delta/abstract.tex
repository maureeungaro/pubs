
\begin{center}
 \vspace{4cm}
 \Large 
 \vspace{4cm}
 
 $\pi^0$ elctroproduction from $\Delta (1232)$ 
 at high momentum transferred with CLAS
 \end{center}


\begin{center}
 \vspace{4cm}
 by Maurizio Ungaro
\end{center}

\begin{center}
 \vspace{2cm}
Version 1.4.1 (2009/05/11)
\end{center}



\specialhead{ABSTRACT}
 

The $N\rightarrow\Delta $ transition has played a major role
for many years
in understanding the structure of the nucleon. The electromagnetic quadrupole 
to dipole ratios $R_{EM} = E_{1+}/M_{1+}$ and $R_{SM} = S_{1+}/M_{1+}$
of the transition are considered  key observables in the study of the dynamics
of baryons.

The object of this analysis is the study of unpolarized 
$\pi^0$ electroproduction
of the $\Delta(1232)$ resonance at high $Q^2$
produced in Hall B at  Jefferson Laboratory. 
The electron beam had an energy of $5.75$ GeV and impunged on a cryogenic  Hydrogen target.
The experiment was performed between September 2001 and January 2002.

The CLAS spectrometer was used to detect the scattered electron
and final state proton, and the $\pi^0$ was reconstructed by the
missing mass technique.

The $\pi^0$ angular distributions were obtained over
the full c.m. angles coverage for a $Q^2$  range  $2$ to $6$ GeV$^2$/c$^2$.
The quantities $R_{EM}$ and $R_{SM}$ 
have been extracted using two methods: a multipole truncation technique that assumes $M_{1+}$ dominance
in the  $\Delta(1232)$ region, and the JANR fit, which is based on the JLAB isobar model.


