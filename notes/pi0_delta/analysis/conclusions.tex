\cia\vspace{-2 cm}
\section{Conclusions}
The differential cross section for the $\pi^0$ electroproduction in 
the $\Delta$ resonance has been measured in the $Q^2$ range $2$ to $6$ GeV$^2$
and $1.1 \le W \le 1.4$ GeV, with full coverage of the $\pi^0$ c.m. angles.
The structure functions $\sigma_T  + \epsilon_L \sigma_L$, $\sigma_{LT}$ and $\sigma_{TT}$ 
have been extracted using the $\phi^*$ dependance of the cross section.

Two calculations of the ratios  $R_{EM}$ and $R_{SM}$ have been presented.

A multipoles truncation fit of the data has been performed using the  $M_{1+}$ dominance and $\ell \le 2$
approximation.
This extraction of the ratio $R_{EM}$ suggest a zero crossing between $Q^2$ of $3$ and $4.0$ GeV$^2$,
while the ration $R_{SM}$ is about $-10\%$ and decreases with $Q^2$.
This result for $R_{EM}$ would prove that the helicity is not conserved in this range of momentum transferred.
The validity of the $M_{1+}$ dominance assumption is questionable, given the fact that the multipoles 
coming from background and other resonances reach values up to $20-40\% $ of $|M_{1+}|$.

A JANR fit that use the JLAB unitarian isobar model has been performed  for $Q^2$ up to $5$ $GeV^2$.
This extraction of the ratio $R_{EM}$ shows a constant negative value of around $-2.5\% $  with the point at $5$ $GeV^2$
consistent with zero.
The JANR fit suggest a smaller $M_{1-}$ amplitude when compared with the previous Hall-C data, suggesting 
that the $P_{11}$ signal is not as strong.






