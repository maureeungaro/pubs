\clearpage\newpage


\section{Electron identification}\label{sec:electron-identification}

In CLAS electro-production experiments
the scattered electron defines the timing of each event,
so it is particularly important to
make sure that their identification is correct and that
there is no contamination from particles such as $\pi^-$.

We consider {\textit candidate electrons} every negative track
that produced a hardware trigger (this trigger condition is ensured by choosing
the first entry in the EVNT bank).
The track is also required to have hit matches in the CLAS \v Cerenkov (CC)\cite{bib:cc},
Drift Chambers (DC)\cite{bib:dc},
Electromagnetic Calorimeter (EC)\cite{bib:ec} and Time of Flight (SC or TOF)\cite{bib:ftof},
and to have time-based reconstruction (positive DC status word in DCPB).

Starting from a candidate electron, we use the following studies, detailed in the following sections,
to defined good electron:

\begin{itemize}
    \item CC $\theta$, $\phi$ and time matching in the detector
    \item EC Threshold
    \item EC Sampling Fraction
    \item Track Coordinates in the EC plane
    \item Minimum Ionizing Particles rejection in the EC
    \item Electromagnetic Shower Shape in the EC
    \item {\textit Number of photo-electrons ($nphe$) in the CC
    This cut is not used anymore for identification,
        but the distributions will be shown for comparison with other analyses of e16 data }
\end{itemize}

For each individual cut we will show four histograms to illustrate its impact:

\begin{itemize}
    \item[a.] Variable distribution when no cut at all is applied.
    \item[b.] Variable distribution when all other pid cuts but this one are applied:
    this helps refining cut parameters.
    For example, when looking at the sampling fraction, applying all other
    cuts to helps clean up the plot and to better estimate the sampling fraction cut.
    We also refer as ``calorimeter cuts'' all cuts but the $nphe$ and EC threshold.
    \item[c.] Variable distribution all the pid cuts are reversed: these should be particles other than electrons.
    This condition can help identifying possible contamination.
    \item[d.] Variable distribution when all cuts, including the one under study, are applied.
\end{itemize}

The statistics and effectiveness of each case is reported in the plots.
Only the relevant plots are reported here.
The complete set of plots can be found on the web~\cite{bib:pi0_resonance_id_electron}.
