\section{Fiducial Cuts}




\subsection{Introduction}

A fiducial cut on electrons is introduced to constrain regions of phase space
where the CLAS response peaks at its maximum and remains rather smooth.
Furthermore, some detector inefficiencies are not perfectly reproduced with GSIM and need to be
removed with dedicated cuts.

Traditionally, the fiducial regions are defined in the lab coordinates of the electron reconstructed $\phi, \theta, p$.
However, it is more natural to define the fiducial regions in the detector coordinates, because
the inefficiencies are caused by tracks near their borders or hardware problems.

Since the former approach has been used in several published CLAS papers, we will keep it both as a reference
and for comparison with the new approach.





\subsection{Traditional cuts on the electron lab coordinates $\phi, \theta, p$}
The fiducial cut in the lab coordinates has been determined during the $\pi^0$ analysis in
the $\Delta(1232)$ region \cite{bib:pi0_Delta}.
For each sector, an empirical cut on $\phi$ is introduced as a function of theta and momentum:
$$
\phi \,\,\le\,\, \Delta\phi \,(\theta, p)
$$
which is aimed to define regions of phase space whose distributions are flat in $\phi$.
After careful studies, and following a common approach between different CLAS experiments, the mathematical
form of the cut depends on 6 parameters  $C_i$
and assumes the form:\vspace{-0.3 cm}
$$
\begin{array}{c c c}
    \\
    \Delta\phi   & = & C_4 \left( \sin (\theta - \theta_{cut}) \right) ^{\,E} \\
    \\
    E        & = & C_3\, p\, ^{C_5} \\
    \\
    \theta_{cut} & = & C_1 + \frac{C_2}{p + C_6}
\end{array}
$$

The $\phi$ vs $\theta$ distribution were divided in 10 different momentum bins from $1.6$ to $4.6$ GeV.
Fig. \ref{fig:fidu_etph} shows one example ($p=1.9-2.2$ GeV) of such distributions.
The $\phi$ distributions are also plotted for $\theta$ slices one degree wide as in Fig. \ref{fig:fidu_ephis}
and the $C_i$ parameters are adjusted empirically.

Table \ref{tab:fid_epars} shows the 6 parameters obtained. Fig. \ref{fig:fidu_e3d} shows
the fiducial cut as a function of $p$, $\theta$ and $\phi$ for sector 1.

\begin{table}[h]
    \begin{center}
        \begin{tabular}{|c|c|c|c|c|c|c|}
            \hline
            Sector & $C_1$ & $C_2$ & $C_3$ & $C_4$ & $C_5$    & $C_6$ \\
            \hline
            1      & 12.0  & 20.0  & 0.32  & 32.0  & 0.416667 & 0.14  \\
            2      & //    & 20.7  & 0.36  & 34.0  & //       & //    \\
            3      & //    & 20.2  & 0.32  & 32.0  & //       & //    \\
            4      & //    & 20.5  & 0.32  & 32.0  & //       & //    \\
            5      & //    & 20.5  & 0.29  & 32.0  & //       & //    \\
            6      & //    & 20.0  & 0.32  & 32.0  & //       & //    \\
            \hline
        \end{tabular}
    \end{center}
    \caption[The 6 parameters for electron fiducial cut for each of the 6 sectors.]
    { The 6 parameters for electron fiducial cut for each of the 6 sectors.
    Only $C_2$, $C_3$, $C_4$ are sector dependent. }
    \label{tab:fid_epars}
\end{table}


\begin{figure}[h]
    \centering
    \includegraphics[width=0.9\textwidth ]{img/electron_tph}
    \caption{$\phi$ versus $\theta$ for sector 1 and $p=1.6-1.9$ GeV after the
    electron ID. Left: before fiducial cut. Right: before fiducial cut
        (box/gray) and after fiducial cut (color contour). }
    \label{fig:fidu_etph}
\end{figure}

\begin{figure}[h]
    \centering
    \includegraphics[width=0.9\textwidth ]{img/electron_tphp}
    \caption{The electron fiducial cut for sector 1 as a function of  $\phi, \theta, p$.
    The cut starting point moves back
    as the momentum increases (and $\theta$ decreases). This causes the cut
    to narrow up with momemtum because electrons are detected near the lower
    edges of the detectors.}
    \label{fig:fidu_e3d}
\end{figure}



\clearpage\newpage


\begin{figure}[ht]
    \centering
    \includegraphics[width=0.98\textwidth ]{img/electron_phis}
    \caption{$\phi$ distributions (sector 3) for different $\theta$ and
        $p=1.9-2.2$ GeV. Black: before fiducial cut. Red: after fiducial cut.
        \v Cerenkov inefficiency (section \ref{sec:cc_eff}) is responsible
        for some irregularities at $\phi = 0$ (for example at
        $\theta = 35.5^0 - 36.5^0$) while drift chambers and time of flight
        inefficiency (section \ref{sec:dc_ineff}) causes other irregularities
        (for example at  $\theta = 42.5^0 - 43.5^0$).}
    \label{fig:fidu_ephis}
\end{figure}


\clearpage\newpage




\subsection{ $\theta$ versus momentum cuts}
Sector 2, 5 and 6 present holes and depletions (mainly because of dead time of flight paddles)
which were taken care of with the
cuts in the $\theta$ vs $p$ plane. An example of such cut is shown in Fig. \ref{fig:fidu_etp5}.

\begin{figure}[h]
 \begin{center}
 \includegraphics[width=0.98\textwidth ]{img/electron_tp5}
  \caption[ $\theta$ versus $p$ for sector 5]
          { $\theta$ versus $p$ for sector 5. Two depletions are clearly visible and cut out.}
 \label{fig:fidu_etp5}
 \end{center}
\end{figure}






\subsection{Cuts on detectors coordinates}

As the tracks swim through the detectors, they are subject to inefficiencies near the sector edges.
This applies to the 3 regions of the drift chambers (DC1, DC2, DC3) and the time of flight detector plane (SC).
An edge cut for the  calorimeter was already applied in the electron ID.

The X vs Y distribution of the tracks in the DCs and the SC is shown in Fig.~\ref{fig:xy_all_planes_s1} for
sector one.
In each sector and each plane, 12 bins in X are defined; the Y distribution for each bin
is fitted with a "tent" function $t(y)$ (described in appendix~\ref{sec:tent_function} ) to select the high efficiency edges.




\begin{figure}[ht]
    \centering
    \includegraphics[width=0.4\textwidth ]{img/plane-DC1_intsector-1}
    \includegraphics[width=0.4\textwidth ]{img/plane-DC2_intsector-1}
    \includegraphics[width=0.4\textwidth ]{img/plane-DC3_intsector-1}
    \includegraphics[width=0.4\textwidth ]{img/plane-SC_intsector-1}
    \caption{The X vs Y distribution of the tracks in the drift chambers and the time
    of flight detector plane (SC) for sector 1. The edges selection algoritm described in the text resulted
    in the black lines, which are the fit of the Y distributions for each X bin.}
    \label{fig:xy_all_planes_s1}
\end{figure}


